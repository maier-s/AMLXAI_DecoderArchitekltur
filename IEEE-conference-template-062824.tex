\documentclass[conference]{IEEEtran}
\IEEEoverridecommandlockouts
% The preceding line is only needed to identify funding in the first footnote. If that is unneeded, please comment it out.
%Template version as of 6/27/2024

\usepackage{cite}
\usepackage{amsmath,amssymb,amsfonts}
\usepackage{algorithmic}
\usepackage{graphicx}
\usepackage{textcomp}
\usepackage{xcolor}
\usepackage{hyperref}
\def\BibTeX{{\rm B\kern-.05em{\sc i\kern-.025em b}\kern-.08em
    T\kern-.1667em\lower.7ex\hbox{E}\kern-.125emX}}
\begin{document}

\title{Transformer Decoder Architektur}

\author{\IEEEauthorblockN{1\textsuperscript{st} Stefan Maier}
\IEEEauthorblockA{\textit{Function and Algorithm} \\
\textit{ZF Lifetec}\\
Alfdorf, Deutschland \\
Stefan.Maier1@zf-lifetec.com}

}

\maketitle
%Literatur aus AMLundXAI.bib einfügen mit 
\bibliographystyle{IEEEtran}



\begin{abstract}
Große Sprache Modelle (engl. Large Language Models) wie Chat GPT basieren auf der Transformer Architektur neuronaler Netze. Die Transformer Architektur besteht dabei aus einem Encoder- sowie Decoder-Element. In der Wissenschaft und Forschung existieren auch Architekturen welche nur einen Teil eines Transformers zur Lösung eines Problems verwenden. In dieser Arbeit wird genauer auf Transformer Architekturen eingegange, welche lediglich aus dem Decoder Element bestehen. Eine weit Verbreitete Decoder Architektur ist die von Chat GPT verwendete GPT (kurz: Generative Pretrained Transformer) Architektur. Es wird auf die Architektur und die weiter Entwicklung der Architektur eingegangen.
\end{abstract}

\begin{IEEEkeywords}
component, formatting, style, styling, insert.
\end{IEEEkeywords}

\section{Einführung}
Die Verarbeitung von natürlicher Sprache (engl. Natural Language Processing. kurz NLP) ist ein wichtiger und großer Bestandteil der künstlichen Intelligenz Forschung. Die Forschungsbereiche decken dabei ein breites Spektrum an Aufgaben ab, wie z.B. Beantwortung von Fragen, Semantische Ähnlichkeit, Text Generierung, Dokumenten Klassifikation o.ä. Die ersten Fortschritte wurde im NLP Bereich bereits durch statistische Hidden Markov Modelle erzielt. Die Leistung solcher statistischen Modelle ist jedoch begrenzt und konnte die Komplexität der natürlichen Sprache nicht geeignet abbilden. Mit den Entwicklungen im Bereich des Deep-Learnings gelangen entscheidende Schritte in der NLP Forschung. Die Transformer Architektur, und das damit verbundene vortrainieren großer Sprachmodelle wie BERT, GPT,T5 oder RoBERTa haben große Fortschritte gebracht, die sich nicht nur auf die NLP Forschung beschränken sondern in der allgemeinen Gesellschaft und Wirtschaft Einzug halten.

\subsection{Ziel der Arbeit}
Das Ziel der Arbeit ist es die Decoder Architektur welche Häufig von generativen Sprachmodellen verwendet wird genauer zu beleuchten. Dabei wird die Decoder-only Architektur anhand des Generative Pretrained Transformers erklärt. Neben dem Generative Pretrained Transformer existieren ebenfalls weitere Modelle die kurz erläutert werden. Weiterhin soll die Arbeit die Anwendungsgebiete sowie Potenziale und Grenzen der Decoder Architektur aufzeigen. Im Ausblick wird auf aktuelle Forschungsfelder im Bereich der Decoder Architekturen eingegangen.
\subsection{Aufbau und Struktur der Arbeit}
Die Arbeit ist wie folgt strukturiert: In Kapitel 2 wird zunächst auf die Grundlagen in Form der Decoder Architektur als Teil der Transformer Architektur eingegangen. Dazu wird die historische Entwicklung, der Aufbau und der Trainingsprozess der Decoder Architektur erläutert und diese vom Encoder Teil der Transformer Architektur abgegrenzt. In Kapitel 3 wird auf unterschiedliche mögliche Anwendungsgebiete aus der Forschung und Entwicklung eingegangen. Aus diesen Erkenntnissen werden in Kapitel 4 die Potenziale und damit einhergehenden Grenzen solcher großen Sprachmodelle eingegangen. Kapitel 5 fasst die Ergebnisse zusammen und gibt einen Ausblick auf die aktuellen Forschungsbereiche.
\section{Decoder Architektur}
Im folgenden Abschnitt wird die Decoder Architektur am Beispiel des Generative Pretrained Transformers (GPT) erläutert. Dabei wird auf die historische Entwicklung, den Aufbau, den Trainingsprozess und die Abgrenzung zur Encoder Architektur eingegangen.
\subsection{Historische Entwicklung der GPT-Decoder Architektur}
Transformer bilden das Grundgerüst moderne großer Sprachmodelle. Diese wurden 2017 in \cite{vaswaniAttentionAllYou2023}  beschrieben. 2018 wurde die erste Version von GPT veröffentlicht, welche auf dem Decoder Part der Transformer Architektur aufbaut. Auf Basis der GPT Modell entstanden weitere Decoder-only Modelle welche die folgende \autoref{fig:Historie} gut veranschaulicht.
\begin{figure}[htbp]
    \centerline{\includegraphics[width=\linewidth]{Bilder/EntwicklungLLM.png}}
    \caption{Historische Entwicklung von Transformer Modellen aus \cite{yangHarnessingPowerLLMs}}
    \label{fig:Historie}
\end{figure}
Die \autoref{fig:Historie} zeigt die Entwicklung unterschiedlicher Transformer Modell bis zum letzten Jahr. Im Rahmen dieser Seminararbeit zeigt Decoder-Only Zweig die Entwicklung von reinen Decoder Architekturen. Damit ist GPT1 eines der ersten Modelle welche auf eine reine Decoder Architekutur setzt. Daraus entsprange viele unterschiedliche Decoder-Only Modelle von Google, Meta oder anderen Forschungseinrichtungen. Die aktuellsten versionen von Llama, GPT-4 oder Bard stellen dabei bis heute die neusten Entwicklungen dar.
\subsection{Aufbau der GPT-Decoder Architektur}
Wie im vorherigen Abschnit beschrieben legt das GPT Modell den Grundstein für die Decoder-only Architektur Entwicklungen. Dies wurde durch das Paper \cite{radfordImprovingLanguageUnderstanding} beschrieben und basiert dabei auf den Grundzügen der Transformer Architekturen welche bereits in Paper \cite{vaswaniAttentionAllYou2023} beschrieben wurden. Die 
folgende Abbildung illustriert dabei den Aufbau einer Decoder Schicht des GPT Modells.
\begin{figure}[htbp]
    \centerline{\includegraphics[width=0.4\linewidth]{Bilder/DecoderArchitekturGPT.png}}
    \caption{Decoder Architektur nach \cite{radfordImprovingLanguageUnderstanding}}
    \label{fig:DecoderArchitektur}
\end{figure}
Die \autoref{fig:DecoderArchitektur} zeigt dabei die Architektur Elemente einer Decoder Schicht. Diese Schichten werden in der ersten GPT Architektur 12 mal hintereinander geschaltet. Jedes dieser Decoder Schichten besteht dabei aus einem Masked Multi-Self Attention Layer, Feed Forward Netzwerken, sowie Add und Layer Normalisation Schicht um die Residual Verbindungen der vorherigen Schicht hinzuzufügen. Die Residual Verbindungen sollen vorallem das Problem der verschwindend Gradienten beheben. Der wichtigste Bestandteil der Decoder Schicht ist dabei das Masked Multi-Self Attention Layer. Das Masked Multi-Self Attention Layer ist dabei eine leichte Abwandlung des Mult Self Attention Layers das auch in Encoder Elementen verwendet wird. Nachffolgende Abbildung zeigt den Aufbau eines Multi-Self Attention Layers.
% Neues Bild
\begin{figure}[htbp]
    \centerline{\includegraphics[width=\linewidth]{Bilder/MultiheadSelfAttention.png}}
    \caption{Multi-Self Attention Layer nach \cite{WieTransformatorenFunktionieren}}
\label{fig:MultiSelfAttention}
\end{figure}
Die Multiplen Heads in der \autoref{fig:MultiSelfAttention} repräsentieren dabei die Anzahl der Köpfe die in der Attention Schicht verwendet werden welche durch die Anzahl an $h$ Köpfe an parallelen Berechnung des Attention Layers bestimmt sind.
zur Berechnung der Ausgabe des Attention Layers werden die Attention Weights berechnet welche bestimen wie wichtig ein Wort für ein anderes Wort ist, was durch nachfolgende Abbildung Visualisiert ist.
% Bild AttentionMechanismus
\begin{figure}[htbp]
    \centerline{\includegraphics[width=\linewidth]{Bilder/AttentionMechanismus.png}}
    \caption{Attention Mechanismus nach \cite{WieTransformatorenFunktionieren}}
\label{fig:AttentionMechanismus}
\end{figure}
Die Färbung in der \autoref{fig:AttentionMechanismus} zeigt dabei die Wichtigkeit der Wörter zueinander. Die daraus berechnete Attention Weights werden mit dem Werten der Wörter multipliziert und über ein Lineares Layer verfeinert, dies ergibt die Ausgabe des Attention Layers. Der wesentliche Unterschied zwischen dem Multi-Self Attention Layer und dem Masked Multi-Self Attention Layer ist dabei die Maskierung der Attention Weights. Die Maskierung sorgt dafür, dass ein Wort nur von vorherigen Wörtern abhängt und damit die Attention einbezieht. Die nachfolgende Abbildung zeigt die Maskierung der Attention Weights mithilfe eine Maskierungsmatrix.
% Bild Maskierung

\begin{figure}[htbp]
    \centerline{\includegraphics[width=\linewidth]{Bilder/MaskierungAttentionWeights.png}}
    \caption{Maskierung der Attention Weights nach \cite{WieTransformatorenFunktionieren}}
\label{fig:Maskierung}
\end{figure}
Die genaue Funktionsweise der einzelnen Schichten aus \autoref{fig:DecoderArchitektur}  wird in \cite{vaswaniAttentionAllYou2023} beschrieben. Abschlißend wird in der GPT Architektur ein Layer je nach Aufgabe angehängt, welche in \autoref{fig:DecoderArchitektur} durch Text Vorhersage oder Aufgaben Klassifizierung repräsentiert wird. Dies kann beispielsweise ein Lineares Layer mit Softmax Aktivierungsfunktion sein, welches die Wahrscheinlichkeiten für das nächste Wort aus dem Vokabular berechnet \cite{WieTransformatorenFunktionieren}.
\subsection{Abgrenzung der Decoder Architektur zur Encoder Architektur}
Die vorgeschlagende Decoder Architektur durch \cite{radfordImprovingLanguageUnderstanding} unterscheidet sich dabei von der Encoder Architektur durch die Verwendung von Masked Multi-Self Attention Layern. Die Encoder Architektur verwendet hingegen Multi-Self Attention Layer ohne Maskierung. Entscheidend ist ebenfalls das die Architektur aus \cite{radfordImprovingLanguageUnderstanding} keine einzeziehung der Encoder Schicht vorsieht und somit im Vergleich zur Decoder Architektur aus \cite{vaswaniAttentionAllYou2023} keine weitere Multi-Self Attention Layer enthält. Die Decoder Architektur ist dabei auf die Generierung von Text ausgelegt und kann somit als autoregressives Modell betrachtet werden. Die Encoder Architektur hingegen ist auf die Verarbeitung von Texten ausgelegt und kann als diskriminatives Modell betrachtet werden.
\subsection{Trainingsprozess der GPT-Decoder Architektur}
Das entscheidende bei der in \cite{radfordImprovingLanguageUnderstanding} beschrieben Decoder Architektur ist der Trainingsprozess der aus zwei Stufen besteht.
\paragraph{unüberwachtes Vortrainieren}
Zum einen das unüberwachte Vortrainieren des om Abschnitt A beschriebenen Modells. Dabei ist die Zielfunktion folgende Likelyhoodfunktion:
\begin{equation}
    {\cal L}_1({\cal U}) = \sum_{i}^{} log P( u_i|u_i-k ... u_i-1;\theta)
    \label{eq:Zielfunktion1}
\end{equation}
k ist dabei das Kontext Fenster und somit wird die Wahrscheinlichkeit des nächsten Wortes auf Basis der vorherigen $k$-Tokens bestimmt. Das erste GPT Modell wurde dabei auf Tokens aus dem BookCorpus dataset vortrainiert. Wie bereits in \autoref{ig:DecoderArchitektur} beschrieben wurden 12 Decoder Schichten verwendet. Für die Attention Heads $h$ wurden 12 Heads verwendet bei 768 Embedding Dimensionen. Für das Feed Forward Netzwerk wurde eine GELU (Gaussian Error Linear Unit) verwendet. Das Training wurde über 100 Epochen mit einer Batchgröße von 64 durchgeführt.
\paragraph{überwachtes Fine-Tuning}
Der zweite Schritt ist das überwachte Fine-Tuning auf die gewünschte Zielaufgabe. Dabei handelt es sich um $m$-Eingabe Tokens, welche in eine Klasse aus $\cal C$ transformiert werden. Die Zielfunktion ist folgende Likelyhoodfunktion:
\begin{equation}
    {\cal L}_2({\cal C}) = \sum_{(x,y)}^{} log P(y|x^1 ... x^m)
    \label{eq:Zielfunktion2}
\end{equation}
Dabei wird die Zielfunktion aus \autoref{eq:Zielfunktion2} mit der Funktion aus \autoref{eq:Zielfunktion1} zu \autoref{eq:Zielfunktion3} kombiniert damit der Trainingsprozess besser konvergiert, sowie generalisiert \cite{radfordImprovingLanguageUnderstanding}.
\begin{equation}
    {\cal L}_3({\cal C}) = {\cal L}_2({\cal C}) + \lambda {\cal L}_1({\cal C})
    \label{eq:Zielfunktion3}
\end{equation}
Da das vortrainierte Modell wie in \autoref{eq:Zielfunktion1} beschrieben auf die vorhersage der nächsten Worte optimiert ist, müssen für das Finetuning die Input Tokens entsprechend der nachfolgenden \autoref{fig:InputTokens} wie in \cite{radfordImprovingLanguageUnderstanding} angepasst werden.
% Bild Input Tokens
\begin{figure}[htbp]
    \centerline{\includegraphics[width=\linewidth]{Bilder/TokenAnpassungGPT1.png}}
    \caption{Anpassung der Input Tokens nach \cite{radfordImprovingLanguageUnderstanding}}
\label{fig:InputTokens}
\end{figure}
Dabei werden strukturierte Daten in eine geordnete Reihenfolge gebracht, um die Eingabe für das Modell zu generieren. \autoref{fig:InputTokens} zeigt, dass für eine Multiple Choice Aufgabe zuerst ein zufällig ausgewähltes Start Token, der Kontext und anschließend die Antwort mit einem zufällig ausgewählten Wert für das Abschluss Token übergeben wird. Für diese spezifische Aufgabe werden anschließend drei Transformer parallel trainiert, um die korrekte Antwort zu vorherzusagen. Für das eigentliche Training wurden die meisten Hyperparameter aus dem Unsuperwised Training übernommen. Lediglich wurde ein Dropout vor der Klassifizierungsschicht hinzugefügt und die Lernrate wurde reduziert. Das Paper aus \cite{radfordImprovingLanguageUnderstanding} zeigt dabei, dass 3 Epochen ausreichen um eine gewünschte Genauigkeit zu erzielen.

\subsection{Weiterentwicklung der GPT-Decoder Architektur}
Die erste GPT Version ist ein Meilenstein in der Entwicklung von Decoder-only Architekturen. Die Weiterentwicklung der GPT Architektur zeigt dabei die Entwicklung von GPT-2, GPT-3 und GPT-4. GPT2 verbesserte die Funktionsweise von großen Sprachmodellen dass diese auf mehrere Aufgaben trainiert werden. Dabei war das Ziel ein Modell zu entwerfen das eine Warscheinlichkeit für eine Output Token Sequenz in Abhängigkeit der Input Tokens sowie der Aufgaben Spezifikation zu erlernen.
\begin{equation}
    P(s_{n-k}, ..., s_n|s_1, ..., s_{n-k-1},task)
    \label{eq:Wahrscheinlichkeit}
\end{equation}
Für das Training das GPT2 Modell wird als Datensatz WebText verwendet welche aus Inhalten kuratierter Websiten besteht. Die Modell struktur ist ähnlich zur ursprünglichen GPT Architektur, jedoch wurde die Anzahl des Kontextfensters von 512 auf 1024 erhöht. Anstatt 12 Decoder-only Layers wurden bei GPT2 48 Layers verwenden mit einer Embedding Dimension von 1600 was zu einer Parametermenge von ca. 1,5 Milliarden Parameter führt. GPT2 zeigt zum ersten mal das ein großes Sprachmodell auf mehrere Aufgaben gleichzeitig trainiert werden kann, sofern der Trainingsdatensatz divers genug ist \cite{radfordLanguageModelsAre}. Während der Fokus bei GPT2 vorallem auf der Zero Shot Performance und das Unsupervised Mulittask Learning liegt, wurde das GPT3 Modell im Fokus des Few Shot Learnings betrachtet. GPT3 erweitert auch hier die Funktionalität von GPT2 erneut und verwendet statt des Masked Multi-Self Attention Layer, Attention Konzepte aus \cite{childGeneratingLongSequences2019} ähnlich der sogenannten Sparse Transformes. Diese ermöglichen es Längere Sequenzen zu verarbeiten und das Training tieferer Modelle zu ermöglichen. Das Paper aus \cite{brownLanguageModelsAre2020} zeigt dass größere Modelle entscheidende Vorteile beim Meta Learning oder Few-Shot Learning haben. So besteht das größte GPT3 Modell aus 175 Milliarden Parametern. Das neueste Modell von OpenAI GPT4 zeigt die Weiterentwicklung von GPT3 und setzt zusätzlich zum Unsupervised Vortraining auf Finetuning über Menschen gesteuertes Feedback mittels Reinforcement Learning. Leider gibt es zum aktuellen Zeitpunkt keine weiteren Details zur Architektur von GPT4 wie in \cite{openaiGPT4TechnicalReport2024} beschrieben. Das Paper in \cite{openaiGPT4TechnicalReport2024} zeigt jedoch anhand der verwendeten Trainingsressourcen, dass das GPT Modell entsprechend groß ist. Zudem wird beschrieben, dass eine Infrastruktur im Rahmen der GPT4 Entwicklung entwickelt wurde um verlässliche die Entwicklung des Modells vorher zu sagen und die Trainingsressourcen zu optimieren.

\section{Anwendungsbereiche}
Die GPT Paper zeigen bereits bei der Model Evaluation auf unterschiedliche Datensätze, die Anwendungsbereiche großer Sprachmodelle. Die Anwendungsbereiche sind dabei vielfältig. In den unterschiedlichen Papern werden bei der Evaluation ebenfalls unterschiedliche NLP Aufgaben betrachtet die für eine Anwendung großer Sprachmodelle in Frage kommen. Es lassen sich 10 Anwendungsbereiche auf Basis von \cite{keary12PraktischeLarge2023} identifizieren, die große Potenziale bieten.
\subsection{Übersetzung}
\cite{jiaoChatGPTGoodTranslator2023} zeigt, dass ChatGPT ein guter Übersetzer ist. Es zeigt sich jedoch auch, dass ChatGPT besser europäische Sprachen übersetzen kann als fernere Sprachen.
\subsection{Erstellung von Inhalten}
Das GPT3 Paper \cite{brownLanguageModelsAre2020} zeigte bereits, dass dieses Nachrichten Artikel generieren kann welche für den Menschen als schwer unterscheidbar gelten. Daher eignenen sich große Sprachmodelle auch für die Erstellung diverser Inhalte. Beispielsweise können Inhalte erstellt werden, die der Ideenfindung oder Inspiration dienlich sein können. Beispiele können hierfür Blogs, Fragebögen oder auch beiträge in sozialen Medien sein \cite{keary12PraktischeLarge2023}.
\subsection{Suchwerkzeug}
Im Anwendungsbereich der Suchwerkzeuge eignet sich ein großes Sprachmodell, wie bereits das GPT2 Paper \cite{radfordLanguageModelsAre} zeigt, eigenen sich Modelle sehr gut um Fragen zu beantworten. Weiterhin kann für eine Schlüsselwort Suche (engl. Keyword Research) verwendet werden um nach die wichtigsten Schlüsselwörter für ein Thema zu erhalten. Damit können SEO-freundliche Inhalte erstellt werden \cite{keary12PraktischeLarge2023}.
\subsection{Virtuelle Assistenten und Kundenbetreuung}
Als Einsatz von Chatbots zeigt auch Zalando in \cite{wuttkeLargeLanguageModel2023}, dass große Sprachmodelle für die Kundenbetreuung eingesetzt werden können. Dabei können Kundenanfragen automatisiert beantwortet werden. Das Sprachmodell unterstützt dabei den potenziellen Kunden bei der Auswahl von Produkten für bestimmte Anlässe zu beraten oder gibt Vorschläge auf basis der persönlichen Präferenzen und Vorlieben.
\subsection{Erkennung und Verhinderung von Cyberangriffen}
Große Sprachmodelle eignen sich unter anderem zur Erkennung von Cyberangriffen SentinalOne sowie Microsoft zeigen wie große Sprachmodelle Datensätze auf Muster hinsichtlich Cyberangriffen scannen um so frühzeitig Angriffe zu erkennen \cite{MicrosoftCopilotSecurity}\cite{SentinelOneUnveilsRevolutionary}.
Mit dem LLM SecPaLM zeigt Google, dass große Sprachmodelle verwendet werden können um das Verhalten von Skripten scannen zu können. Damit muss der Benutzer die Software nicht zwangsläufig in einer Sandbox ausführen um zu prüfen ob die Software schädlich ist \cite{keary12PraktischeLarge2023}.
\subsection{Code-Entwicklung}
Ein weiterer wichtiger Anwendungsfall ist die Unterstützung von großen Sprachmodellen bei der Code-Entwicklung. LLM's können dabei Code Abschnitte schreiben, Dokumentation generieren oder bei der Fehlersuche während des Programmierens unterstützen. \cite{keary12PraktischeLarge2023}
\subsection{Transkription}
Im Paper von GPT4 \cite{openaiGPT4TechnicalReport2024} wird auf die Multimodalität eingegangen und gezeigt wie gut große Sprachemodelle beispielsweise Bilder verarbeiten können. Diese Fähigkeit kann dazu genutzt werden um Transkritpionen anzufertigen \cite{keary12PraktischeLarge2023}.
\subsection{Marktforschung}
In der Marktforschung müssen viele Datensätze analysiert und ausgewertet werden. So kann ein großes Sprachmodell beispielsweise bei der Marktforschung unterstützen indem es Zusammenfassungen erstellt, Trends erkennt oder Marktlücken identifitiert \cite{keary12PraktischeLarge2023}.
\subsection{Vertriebsautomatisierung}
Große Sprachmodelle können auch im Vertriebswesen verwendete werden um Vertribsprozesse zu automatisieren. \cite{keary12PraktischeLarge2023} zeigt, dass LLM's bei der Lead-Generierung, Pflege, Personalisierung, Qualifizierung, Bewertung sowie Lead Prognose unterstützen kann, indem es entsprechende Lead Datensätze auswertet.
\subsection{Stimmungsanalyse}
Ein weiterer wichtiger Anwendungsbereich ist in der Stimmungsanalyse. Große Sprachmodelle können dabei helfen die Stimmung von Texten zu analysieren und so beispielsweise die Stimmung von Kunden zu analysieren \cite{keary12PraktischeLarge2023}. Mit der Stimmungsanalyse können Unterhnehmen die assozierten Wörter mit dem Unterhnehmen aus Kommentaren herausfiltern und erhalten ein Bild über die Marke des Unternehmens. Abhängig davon lässt sich die Marke entsprechend der Stimmung verbessern \cite{keary12PraktischeLarge2023}.

\section{Potenzial und Grenzen von Decoder Architekturen}
Der vorherige Abschnitt zeigt das breite Anwendungsspektrum von Decoder Architekturen. Da sich Decoder Architekturen stetig weiter entwickeln werden sich die Anwendungsfelder ebenfalls erweitern. Nach \cite{keary12PraktischeLarge2023} wird das Marktpotenzial generativer KI's bis 2032 auf bis zu 1,3 Billionen Dollar geschätzt. Neben den Potenzialen gibt es jedoch auch Grenzen die bei der Verwendung von Decoder Architekturen beachtet werden müssen. Diese wurden in \cite{radfordLanguageModelsAre} und \cite{openaiGPT4TechnicalReport2024} für das GPT Modell genauer beschrieben.


\section{Fazit}
\subsection{Aktuelle Forschungbereiche}
\subsection{Ausblick}





\bibliography{AMLundXAI}

\end{document}
